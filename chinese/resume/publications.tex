%-------------------------------------------------------------------------------
%	SECTION TITLE
%-------------------------------------------------------------------------------
\cvsection{学术论文}

%-------------------------------------------------------------------------------
%	CONTENT
%-------------------------------------------------------------------------------
\begin{cventries}

%---------------------------------------------------------
  \cventry
    {Ce Gao, Rui Ren and Hongming Cai} % Award
    {GAI: A Centralized Tree-Based Scheduler for Machine Learning Workload in Large Shared Cluster} % Event
    {ICA3PP'18 (CCF-C)} % Location
    {2018.11} % Date(s)
    {
      \begin{cvitems} % Description(s)
        \item {
          本文分析了机器学习模型的训练,识别了训练过程中的短板效应:与 CPU 训练相比,GPU 训练需要更高的网络带宽。这一观察启发了 GAI 的设计,GAI 是一个集中式的调度器,用于机器学习工作负载。它依赖于两种技术:1)树型结构。该结构分层存储集群信息,实现多层调度。2)扩展良好的优先级算法。我们全面考虑了模型培训工作的多个优先级,以支持资源退化和抢占。在 Kubernetes、Kubeflow 和 TensorFlow 上实现了 GAI 的原型。它是通过一个模拟器和一个真正的基于云的集群进行评估的。结果表明,在 DL 模型上,调度吞吐量提高了28\%,训练收敛速度提高了21\%
        }
      \end{cvitems}
    }

%---------------------------------------------------------
  \cventry
    {Xinyuan Huang, Amit Saha, Debojyoti Dutta and Ce Gao} % Award
    {Kubebench: A Benchmarking Platform for ML Workloads} % Event
    {IEEE AI4I'18} % Location
    {2018.9} % Date(s)
    {
    }

\end{cventries}
