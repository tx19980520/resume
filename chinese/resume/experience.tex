%-------------------------------------------------------------------------------
%	SECTION TITLE
%-------------------------------------------------------------------------------
\cvsection{工作经历}


%-------------------------------------------------------------------------------
%	CONTENT
%-------------------------------------------------------------------------------
\begin{cventries}

%---------------------------------------------------------
  \cventry
    {合作研究} % Job title
    {才云科技} % Organization
    {中国上海} % Location
    {2017 年 11 月至今} % Date(s)
    {
      \begin{cvitems} % Description(s) of tasks/responsibilities
        \item 研究 Kubernetes 上对于机器学习基准测试的系统 \href{https://github.com/kubeflow/kubebench}{\bf kubeflow/kubebench},研究成果发表在 IEEE AI4I'18 会议,在 KubeCon China 2018 发表时长 30 分钟的演讲。
        \item 实现和维护 Kubeflow 在 Jupyter 上的内核项目 \href{https://github.com/caicloud/ciao}{\bf caicloud/ciao},支持从 Jupyter 中发起分布式机器学习训练任务。
        \item 为 \href{https://github.com/kubeflow/kubeflow}{Kubeflow} 社区维护 TensorFlow 分布式训练支持 \href{https://github.com/kubeflow/tf-operator}{\bf kubeflow/tf-operator} 和超参数训练系统 \href{https://github.com/kubeflow/katib}{\bf kubeflow/katib}。
        \item 研究分布式机器学习任务在大规模机器集群上的调度,研究成果发表在 ICA3PP'18 会议。
      \end{cvitems}
    }

%---------------------------------------------------------
  \cventry
    {Google Summer of Code 学生} % Job title
    {Processing 基金会} % Organization
    {中国上海} % Location
    {2017 年 3 月至 2017 年 9 月} % Date(s)
    {
      \begin{cvitems} % Description(s) of tasks/responsibilities
        \item 参与 Google Summer of Code 活动,本次 GSoC 申请的接收率为 6\%(1318/20651)
        \item 基于 R 语言在 Java 虚拟机上的解释器,设计并实现了 Processing 在 R 语言模式 \href{https://github.com/gaocegege/Processing.R}{\bf Processing.R},获得 94 stars,成为本次编程之夏 star 最多的项目
      \end{cvitems}
    }

\end{cventries}
