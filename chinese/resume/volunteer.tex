%-------------------------------------------------------------------------------
%	SECTION TITLE
%-------------------------------------------------------------------------------
\cvsection{其他经历}


%-------------------------------------------------------------------------------
%	CONTENT
%-------------------------------------------------------------------------------
\begin{cventries}

%---------------------------------------------------------
  % \cventry
  %   {导师} % Job title
  %   {Google Code-In 2018} % Organization
  %   {中国上海} % Location
  %   {2018 年 9 月至今} % Date(s)
  %   {
  %     \begin{cvitems}
  %       \item 制定难度符合高中生水平的任务,指导来自全球的高中生在 coala 社区中参与 Google Code-In 开源竞赛
  %     \end{cvitems}
  %   }

  \cventry
    {MOOC \& Open Source \& web 组员} % Job title
    {上海交通大学东岳网络工作室} % Organization
    {中国上海} % Location
    {2019 年 11 月至今} % Date(s)
    {
      \begin{cvitems}
        \item 维护东岳网络工作室的技术博客以及知乎专栏:\href{https://zhuanlan.zhihu.com/dongyue}{东岳网络工作室团队}
        \item 维护上海交通大学 XeLaTeX 学位论文模板:\href{htts://github.com/sjtug/sjtuthesis}{SJTUThesis}
        \item 进行定期技术分享,组织成员进行开源社区的贡献活动
        \item \href{https://tongqu.me}{同去网}维护
      \end{cvitems}
    }

  \cventry
    {部长} % Job title
    {上海交通大学电院学生会网络部} % Organization
    {中国上海} % Location
    {2018 年 9 月至 2020 年 2 月} % Date(s)
    {
      \begin{cvitems}
        \item 维护电源学生会相关活动网站:\href{https://zhuanlan.zhihu.com/dongyue}{东岳网络工作室团队}
        \item 进行定期技术分享,组织成员进行开源社区的贡献活动
        \item 负责电院信息汇总小程序开发(小程序名seiee信息汇总小助手)
      \end{cvitems}
    }

\end{cventries}
