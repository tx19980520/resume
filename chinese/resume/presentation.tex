%-------------------------------------------------------------------------------
%	SECTION TITLE
%-------------------------------------------------------------------------------
\cvsection{演讲与分享}


%-------------------------------------------------------------------------------
%	CONTENT
%-------------------------------------------------------------------------------
\begin{cventries}

%---------------------------------------------------------
  \cventry
    {KubeCon China 2018}
    {\href{http://sched.co/FvLV}{对 Kubeflow 上的机器学习工作负载做基准测试}}
    {中国上海} % Location
    {2018 年 11 月} % Date(s)
    {
      \begin{cvitems} % Description(s)
        \item {
          在本次演讲中我们介绍基于 Kubeflow 的开源基准化工具 Kubebench,其帮助我们通过自动化和一致的规范,更好的理解 Kubernetes 上的 ML 工作量的性能特征。我们还说明我们可以怎样利用来自学术界和工业界的其他基准化成就,如 MLPerf 和 Dawnbench。
        }
      \end{cvitems}
    }

  \cventry
    {统计之都 2018 年 Meetup}
    {\href{https://docs.google.com/presentation/d/1ED24TCnlBVzyJz0aCEAtXQQh0_W1RKSeapP3QZ0fTKA/edit?usp=sharing}{Kubeflow: Run ML workloads on Kubernetes}}
    {中国上海} % Location
    {2018 年 7 月} % Date(s)
    {
    }

  \cventry
    {第十届中国 R 语言会议}
    {\href{http://slides.com/gaocegege/processing-r}{
      Processing.R: 使用 R 语言实现新媒体艺术作品}}
    {中国上海} % Location
    {2017 年 12 月} % Date(s)
    {
      \begin{cvitems} % Description(s)
        \item {
          本次演讲介绍了 Processing.R。通过这一项目,用户可以利用 R 语言进行新媒体艺术作品的创作。
        }
      \end{cvitems}
    }

  \cventry
    {2017 年东岳技术分享}
    {\href{https://docs.google.com/presentation/d/1ylRT4VvydWbR7SyTQzNZOLpkXtgSZJiEl5nmXY1KuJw/edit?usp=sharing}{Processing.R 设计,实现与使用}}
    {中国上海} % Location
    {2017 年 7 月} % Date(s)
    {
    }

  \cventry
    {2016 年东岳技术分享}
    {\href{https://docs.google.com/presentation/d/1Ru4Dm9TLoyxnJgFqvsCHrb82VT622H-zBSgAe1vJL44/edit?usp=sharing}{Docker 入门介绍}}
    {中国上海} % Location
    {2016 年 7 月} % Date(s)
    {
    }

%---------------------------------------------------------
\end{cventries}
